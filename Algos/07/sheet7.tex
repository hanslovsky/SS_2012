\documentclass[a4paper]{article}

%\usepackage{texments}
\usepackage{graphicx}
\usepackage{caption}
\usepackage{subcaption}
\usepackage{tabularx}
\usepackage{booktabs}
\usepackage[table]{xcolor}
\usepackage{minted}
\usepackage{amsmath}
\usepackage{amsfonts}

\usepackage[left=2.5cm, right=2.5cm, bottom=2.5cm,footskip=.5cm, top=2.5cm]{geometry}


\usepackage[utf8]{inputenc}
\usepackage[T1]{fontenc}
\usepackage[ngerman]{babel}

\definecolor{bg}{rgb}{0.950,.95,0.95}
\newminted{pycon}{bgcolor=bg,linenos=false,tabsize=4}

\begin{document}

%%%%%%%%%%%%%%%%%%%%%%%%%%%%%%%%%%%%%%%
%%% 6.1                             %%%
%%%%%%%%%%%%%%%%%%%%%%%%%%%%%%%%%%%%%%%

\section*{Aufgabe1}
Sei $K_n$ eine Menge von Schlüsseln der Länge $n$ mit gleichem Hashwert $C_n$, also $hhash(v)=C_n$, für alle $v\in K_n$ mit festem $C_n\in\mathbb{N}$. Für ein $v\in K_n$ sei $v_i=ord(v[i])$ mit $i\in\{0,1,\dots,n-1\}$. Sei nun $x,y\in K_2$, also zwei beliebige Schlüssel der Länge $2$ mit gleichem Hashwert $C_2$. Sei $z=xy$ ein Schlüssel der Länge $4$ zusammengesetzt aus $x$ und $y$. Für ein $v\in K_2$ gilt:
\begin{equation}
hhash(v)=23v_0 + v_1.\label{eq:hash1}
\end{equation} 
Es folgt:
\begin{align}
hhash(z) & = 23(23hhash(x)+y_0) + y_1\\
& = 23^2hhash(x)+23y_0+y_1\\
& = 23^2hhash(x)+hhash(y)\\
& = 23^2C_2+C_2.
\end{align}
Das heißt jede Kombination von zwei Schlüsseln aus $K_2$ ergibt einen Schlüssel der Länge $4$, die alle den gleichen Hashwert haben. Um $16$ Schlüssel der Länge $4$ mit gleichem Hashwert zu finden, muss man also lediglich $4$ Schlüssel der Länge $2$ finden mit gleichem Hashwert ($4\cdot4=16$ Kombinationen).

Für zwei beliege Elemente $x,y\in K_2$ gilt wegen~\eqref{eq:hash1}
\begin{equation}
y_1=23(x_0-y_0)+x_1.\label{eq:hash2}
\end{equation}
Es sollen keine 'komischen' Zeichen vorkommen, also alle Unicodewerte sollen in $\{32,33,\dots,127\}$ liegen. Es wird nun ein Schlüssel ($a$) der Länge zwei fest vorgegeben und drei weitere Schlüssel ($b,c,d$) der Länge $2$ gesucht mit gleichem Hashwert.  Das intuitivste ist $b_0=a_0-1$, $c_0=a_0-2$ und $d_0=a_0-3$ zu setzen und dann mit~\eqref{eq:hash2} die Werte $b_1$, $c_1$ und $d_1$ berechnen. Damit alle Unicodewerte in $\{32,33,\dots,127\}$ liegen, kann man zum Beispiel $a_0=35$, $a_1=32$ setzen. Tabelle \ref{tab:unicode} zeigt diese $4$ Schlüssel und Tabelle \ref{tab:string} zeigt die zugehörigen Zeichen. Die $16$ Schlüssel der Länge $4$ sind im File \emph{collisions.txt}.


\begin{table}[H]
\centering
\begin{tabular}{|c|c|c|c|c|c|c|c|}
\hline
$a_0$ & $a_1$ & $b_0$ & $b_1$ & $c_0$ & $c_1$ & $d_0$ & $d_1$ \\\hline
$35$ & $32$ & $34$ & $55$ & $33$ & $78$ & $32$ & $101$
\end{tabular}
\caption{UnicodeWerte der Schlüssel}
\label{tab:unicode}
\end{table}


\begin{table}[H]
\centering
\begin{tabular}{|c|c|c|c|}
\hline
$a$ & $b$ & $c$ & $d$ \\\hline
'\# '& '"7' & '!N' & ' e'
\end{tabular}
\caption{UnicodeWerte der Schlüssel}
\label{tab:string}
\end{table}




\end{document}
