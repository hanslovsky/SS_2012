\documentclass[a4paper]{article}

%\usepackage{texments}
\usepackage{graphicx}
\usepackage{caption}
\usepackage{subcaption}
\usepackage{tabularx}
\usepackage{booktabs}
\usepackage[table]{xcolor}
\usepackage{minted}
\usepackage{amsmath}

\usepackage[left=2.5cm, right=2.5cm, bottom=2.5cm,footskip=.5cm, top=2.5cm]{geometry}


\usepackage[utf8]{inputenc}
\usepackage[T1]{fontenc}
\usepackage[ngerman]{babel}

\definecolor{bg}{rgb}{0.950,.95,0.95}
\newminted{pycon}{bgcolor=bg,linenos=false,tabsize=4}

\begin{document}


\section*{Aufgabe 2}
\subsection*{(2 a)}
Da die Punkte gleichverteilt sind, ist die Verteilung der Punkte mit Radius $r$ gegeben durch das Volumenverhältnis 
\begin{equation}
F(r)=\frac{\pi r^2}{\pi}.
\end{equation}
Die Dichtefunktion ist daher
\begin{equation}
f(r)=\frac{dF(r)}{dr}=2r.
\end{equation}
Nun sucht man Intervallgrenzen $r_j$, so dass Punkte im Intervall $[r_j,r_{j+1})$ auf den Index $j$ abgebildet werden. Die Intevallgrenzen sollen so gewählt werden dass im Schnitt in allen Intevallen gleich viele Punkte sind.
Die Anzahl der Punkte im Bucket $i$ ist gegeben durch
\begin{equation}
N_i = \int_{r_i}^{r_{i+1}}2r dr = r_{i+1}^2-r_i^2.\label{eq:N}
\end{equation}
Es muss also gelten $N_i=N_{i+1}$. Mit~\eqref{eq:N} folgt daraus für die Intervallgrenzen $r_{i+2}^2=2r_{i+1}^2-r_i^2$. Daraus folgt $r_i=\sqrt{i}\cdot r_1$. Diese Intervallgrenzen werden realisiert indem man die Indexfunktion
\begin{equation}
index_{quad}(r)=\lfloor r^2M\rfloor
\end{equation}
wählt.

\subsection*{(2b)}
Die Funktion testUniformity gibt $\tau$ zurück. Für die Arraygrößen $10^2,10^3,10^4,10^5,10^6$ und M=$2,4,8,20,100$ wurde getestet ob $\tau>3$ (nicht gleichverteilt). Die Indexfunktion $\lfloor r^2M\rfloor$ besteht alle $25$ Tests, die Indexfunktion $\lfloor rM\rfloor$ besteht nur $1-2$ von 25 Tests.


See figure~\ref{fig:trees} for trees.

\begin{figure}
  \begin{subfigure}[b]{0.5\textwidth}
    \centering
    \includegraphics[width=6cm]{tree1.png}
    \caption{first tree, build from string '2+g*3'}
  \end{subfigure}
  \begin{subfigure}[b]{0.5\textwidth}
    \includegraphics[width=6cm]{tree2.png}
    \caption{second tree, build from string '2*4*(3+(4-7)*8)-(1-6)'}
  \end{subfigure}
  \caption{Trees build from string expressions by algorithm described in 2a}
  \label{fig:trees}
\end{figure}


\subsection*{(2c)}


\begin{minted}{python}
dic = {'+': operator.add, '-': operator.sub, "*": operator.mul, 
       "/": operator.div, "0":0, "1":1, "2":2, "3":3, "4":4, "5":5, "6":6,
       "7":7, "8":8, "9":9}


def calculateFromTree(tree):
    """ calculate expression from tree
    """
    if tree.left.left != None: # left child is operator
        calculateFromTree(tree.left)
    if tree.right.left != None: #right child is operator
        calculateFromTree(tree.right)

    #both children are numbers -> do operation
    if tree.left.left == None and tree.right.left == None:                    
        tree.key = dic[tree.key](int(tree.left.key), int(tree.right.key))
        tree.left = None
        tree.right = None
    return tree.key


def evalString(string):
    """ evaluate expression given as string
    build tree from string, then calculate expression from tree
    no checking that input string is correct...
    """
    tree = buildTree(string)
    result = calculateFromTree(tree)

    return result
\end{minted}


%\inputminted{python}{1.2.py}



\end{document}
