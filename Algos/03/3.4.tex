\documentclass[a4paper]{article}

%\usepackage{texments}
\usepackage{tabularx}
\usepackage{booktabs}
\usepackage[table]{xcolor}
%\usepackage{minted}

\usepackage[left=2.5cm, right=2.5cm, bottom=2.5cm,footskip=.5cm, top=2.5cm]{geometry}


\usepackage[utf8]{inputenc}
\usepackage[T1]{fontenc}
\usepackage[ngerman]{babel}

\definecolor{bg}{rgb}{0.950,.95,0.95}
%\newminted{pycon}{bgcolor=bg,linenos=false,tabsize=4}

\begin{document}

\section*{Aufgabe 4}
\subsection*{(4 a)}

\begin{table}[h]
\caption{Vorbedingungen und Nachbedingungen für Deque Funktionen}
\label{bookmarksavepage}\centering
\begin{tabularx}{\textwidth}{X|X|X}
\toprule
Funktion & Vorbedingungen & Nachbedingungen \\\midrule
Konstruktor q.Deque(N) & type(N) == int & $q\,\in\,Deque$ \\
& N $\geq$ 1 & Kapazität von q == N \\\midrule
q.size() & $q\,\in\,Deque$ & q wird durch die Funktion nicht verändert \\
& & Rückgabewert entspricht der tatsächlichen Größe \\\midrule
q.capacity() & $q\,\in\,Deque$ & q wird durch die Funktion nicht verändert \\
& & Rückgabewert entspricht der tatsächlichen Kapazität \\\midrule
q.push(x) & $q\,\in\,Deque$ & Das letzte Arrayelemnt hat den Wert x \\
& x ist ein gültiger Typ und Wert (für die Implementation in Python spielt das keine Rolle, da eine Liste Variablen unterschiedlicher beliebiger Typen halten kann) & falls die Größe nicht der Kapazität entspricht, wird sie um eins erhöht \\\midrule
x = q.popLast() & $q\,\in\,Deque$ & Länge von q wird um eins vermindert \\
& q ist nicht leer & x hat den Wert des entfernten Elements \\\midrule
x = q.popFirst() & $q\,\in\,Deque$ & Länge von q wird um eins vermindert \\
& q ist nicht leer & x hat den Wert des entfernten Elements \\\bottomrule



\end{tabularx}
\end{table}



\end{document}
