\documentclass[a4paper]{article}

%\usepackage{texments}
\usepackage{tabularx}
\usepackage{booktabs}
\usepackage[table]{xcolor}
\usepackage{minted}

\usepackage[left=2.5cm, right=2.5cm, bottom=2.5cm,footskip=.5cm, top=2.5cm]{geometry}


\usepackage[utf8]{inputenc}
\usepackage[T1]{fontenc}
\usepackage[ngerman]{babel}

\definecolor{bg}{rgb}{0.950,.95,0.95}
\newminted{pycon}{bgcolor=bg,linenos=false,tabsize=4}

\begin{document}

\section*{Aufgabe 2}
\subsection*{(2 c)}
Es gilt
\begin{equation}
\bar{s}_{2n}=\frac{s_n}{\sqrt{2+\sqrt{4-s_n^2}}}\qquad und\qquad
s_{2n}=\sqrt{2-\sqrt{4-s_n^2}}.
\end{equation}
Es folgt
\begin{equation}
\frac{s_{2n}^2}{\bar{s_{2n}}^2}=\frac{(2-\sqrt{4-s_n^2})(2+\sqrt{4-s_n^2})}{s_n^2}=\frac{4-(4-s_n^2)}{s_n^2}=1,
\end{equation}
wobei die Binomische Formel, $(a+b)(a-b)=a^2-b^2$, benutzt wurde.\\
Es gilt
\begin{equation}
\bar{t_{2n}}=\frac{2t_n}{\sqrt{4+t_n^2}+2}\qquad und \qquad 
t_{2n}=\frac{2}{t_n}\left(\sqrt{4+t_n^2}-2\right).
\end{equation}
Es folgt
\begin{equation}
\frac{t_{2n}}{\bar{t_{2n}}}=\frac{2(\sqrt{4+t_n^2}-2)(\sqrt{4+t_n^2}+2)}{2t_n^2}=1.
\end{equation}
Also gilt $\bar{s_{2n}}=s_{2n}$ und $\bar{t_{2n}}=t_{2n}$.
%\subsection*{(2b)}

%\begin{pyconcode}

%\end{pyconcode} 


%\inputminted{python}{1.2.py}



\end{document}
