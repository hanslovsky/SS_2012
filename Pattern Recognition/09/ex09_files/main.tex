\documentclass[a4paper,11pt]{article}
\usepackage[left=1.5cm,right=1.5cm,top=2.0cm,bottom=2.8cm]{geometry}

\usepackage[ngerman,american]{babel}
\usepackage{graphicx}
\usepackage{amsmath}
\usepackage{amsthm}
\usepackage{amssymb}
\usepackage{listings}
\usepackage{epstopdf}
\usepackage{sidecap}
% \usepackage[FIGTOPCAP]{subfigure}
\usepackage{float}
\usepackage{color}
\usepackage[usenames,dvipsnames]{xcolor}
%\usepackage{empheq}
\usepackage[footnotesize,bf]{caption}
\usepackage{subcaption}
%\usepackage[framed,numbered,autolinebreaks,useliterate]{mcode}

\theoremstyle{definition}
\newtheorem{defi}{Definition}
\theoremstyle{plain}
\newtheorem{theo}[defi]{Theorem}
\theoremstyle{remark}
\newtheorem{remark}{Remark}

\providecommand{\abs}[1]{\lvert#1\rvert}
\providecommand{\norm}[1]{\lVert#1\rVert}

\renewcommand{\vec}[1]{\boldsymbol{#1}}

\title{Exercise 9}
\author{Philipp Hanslovsky, Robert Walecki}

\begin{document}


\lstloadlanguages{R} 
\lstset{language=R,
   %keywords={break,case,catch,continue,else,elseif,end,for,function,
   %   global,if,otherwise,persistent,return,switch,try,while},
   basicstyle=\ttfamily,
   keywordstyle=\color{blue},
   commentstyle=\color{green!40!black},
   stringstyle=\color{red},
   numbers=left,
   numberstyle=\tiny\color{white!50!black},
   stepnumber=1,
   numbersep=10pt,
   backgroundcolor=\color{white},
   tabsize=4,
   showspaces=false,
   showstringspaces=false,
   frame=single}


\maketitle

\section*{9.1.1}
\begin{figure}[H]
\centering
\includegraphics[width=0.7\textwidth]{scatter.pdf}
\caption{scatter plot of the distribution of pixels in color space}
\label{fig:scat}
\end{figure}

\section*{9.1.2}
\begin{figure}[H]
\centering
\begin{subfigure}{0.45\textwidth}
\includegraphics[width=\textwidth]{centers.pdf}
\caption{RGB}
\end{subfigure}
\hfill
\begin{subfigure}{0.45\textwidth}
\includegraphics[width=\textwidth]{centersXY.pdf}
\caption{RGB components of RGBXY clusters}
\end{subfigure}
\caption{scatter plot with four clusters}
\label{fig:clus}
\end{figure}

\begin{figure}[H]
\centering
\includegraphics[width=\textwidth]{compression.pdf}
\caption{left column: original image, center column: compression with RGB feature space ($K\in\{4,16,64\}$), right column: compression with RGBXY feature space ($K\in\{4,16,64\}$)}
\label{fig:compress}
\end{figure}

\section*{9.1.4}
As expected the RGB components of RGBXY clustering differ from the result for RGB clustering (see fig \ref{fig:clus}). The pixels of a single cluster are more spread out in RGB space and the clusters overlap. It becomes most obvious for the light blue cluster, which cannot be recognized as a cluster, when only looking at RGB space and not at RGBXY. The result for image compression is seen in fig \ref{fig:compress}: For four clusters, there are three greenish patches, that make the spatial clustering obvious. Patches like that occur for a larger number of clusters as well, however, not as obvious.









\end{document}
