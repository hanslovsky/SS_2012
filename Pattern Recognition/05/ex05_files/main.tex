\documentclass[a4paper,11pt]{article}
\usepackage[left=1.5cm,right=1.5cm,top=2.0cm,bottom=2.8cm]{geometry}

\usepackage[ngerman,american]{babel}
\usepackage{graphicx}
\usepackage{amsmath}
\usepackage{amsthm}
\usepackage{amssymb}
\usepackage{listings}
\usepackage{epstopdf}
\usepackage{sidecap}
% \usepackage[FIGTOPCAP]{subfigure}
\usepackage{float}
\usepackage{color}
\usepackage[usenames,dvipsnames]{xcolor}
%\usepackage{empheq}
\usepackage[footnotesize,bf]{caption}
\usepackage{subcaption}
%\usepackage[framed,numbered,autolinebreaks,useliterate]{mcode}

\theoremstyle{definition}
\newtheorem{defi}{Definition}
\theoremstyle{plain}
\newtheorem{theo}[defi]{Theorem}
\theoremstyle{remark}
\newtheorem{remark}{Remark}

\providecommand{\abs}[1]{\lvert#1\rvert}
\providecommand{\norm}[1]{\lVert#1\rVert}

\renewcommand{\vec}[1]{\boldsymbol{#1}}

\title{Exercise 5}
\author{Philipp Hanslovsky, Robert Walecki}

\begin{document}

\lstset{language=Matlab,
   keywords={break,case,catch,continue,else,elseif,end,for,function,
      global,if,otherwise,persistent,return,switch,try,while},
   basicstyle=\ttfamily,
   keywordstyle=\color{blue},
   commentstyle=\color{green!40!black},
   stringstyle=\color{red},
   numbers=left,
   numberstyle=\tiny\color{white!50!black},
   stepnumber=1,
   numbersep=10pt,
   backgroundcolor=\color{white},
   tabsize=4,
   showspaces=false,
   showstringspaces=false,
   frame=single}


\maketitle
\section*{5.1.1}
\begin{align*}
&\boldsymbol{x} &=& (x_1, x_2)^T \\
&k(\boldsymbol{x}, \boldsymbol{z}) &=& (\boldsymbol{x}^T\boldsymbol{z})^3 \\
&                                  &=& x_1^3z_1^3+3x_1^3z_1^2x_2z_2+3x_1z_1x_2^2z_2^2+x_2^3z_2^3 \\
\Rightarrow &\boldsymbol{\Phi}(\boldsymbol{x}) &=& (x_1^3, \sqrt{3}x_1^2x_2, \sqrt{3}x_1x_2^2, x_2^3)^T \\
&\Phi: &&\mathbb{R}^2 \mapsto \mathbb{R}^4
\end{align*}


\section*{5.1.2}
\begin{align}
k(\boldsymbol{x},\boldsymbol{z}) &= c*k_1(\boldsymbol{x},\boldsymbol{z}) &c=const \label{eq:lin} \\
k(\boldsymbol{x},\boldsymbol{z}) &= \exp{(k_1(\boldsymbol{x},\boldsymbol{z})}& \label{eq:exp} \\
k(\boldsymbol{x},\boldsymbol{z}) &= k_1(\boldsymbol{x},\boldsymbol{z})+k_2(\boldsymbol{x},\boldsymbol{z})& \label{eq:sum} \\
k(\boldsymbol{x},\boldsymbol{z}) &= k_1(\boldsymbol{x},\boldsymbol{z})\cdot k_2(\boldsymbol{x},\boldsymbol{z})& \label{eq:prod} \\
\end{align}

\begin{align}
&k(\boldsymbol{x},\boldsymbol{z})& = &\exp{\left(-\frac{\left(\boldsymbol{x}-\boldsymbol{z}\right)^2}{2\sigma^2}\right)}& \\
&                                & = &\prod_{i=1}^3 \exp{\left(-\frac{k_i}{2\sigma^2}\right)}& \\
&k'(\boldsymbol{x}, \boldsymbol{z})&=& \boldsymbol{x}^T\boldsymbol{z} &=& \boldsymbol{\Phi}^T\boldsymbol{\Phi}\\
&k_1 &=& k'(\boldsymbol{x}, \boldsymbol{x})& \\
&k_2 &=& k'(\boldsymbol{z}, \boldsymbol{z})& \\
&k_3 &=& k'(\boldsymbol{x}, \boldsymbol{z})& \\
&\Phi:  (x_1,\dots,x_n) \mapsto (x_1,\dots,x_n)& \\
\end{align}

$k_i$ are valid kernels as $k'$ can be expressed by a scalar product of a mapping of $\boldsymbol{x}$ and $\boldsymbol{z}$. According to equation \ref{eq:lin} $\tilde k_i = -\frac{k_i}{2\sigma^2}$ are valid kernels as well with $c=-\frac{1}{2\sigma^2}$. From equation \ref{eq:exp} we can see that $\exp\left(\tilde k_i\right)$ are valid kernels as well. Finally a product of valid kernels results in a valid kernel (equation \ref{eq:prod}).

\clearpage
\lstinputlisting[language=Matlab, caption={Gaussian kernel response in Matlab}, label = {lst:response}]{gaussian_kernel.m}
\vspace{50pt}
\begin{figure}[H]
\centering
\includegraphics[width=\textwidth]{images/response.eps}
\caption{Responses for various values of $\sigma$ (see plot titles).}
\label{fig:response}
\end{figure}

\begin{figure}[H]
\centering
\begin{subfigure}{0.49\textwidth}
\centering
\includegraphics[width=\textwidth]{images/surf_10.pdf}
\end{subfigure}
\hfill
\begin{subfigure}{0.49\textwidth}
\centering
\includegraphics[width=\textwidth]{images/surf_15.pdf}
\end{subfigure}
\caption{surface plots for $\sigma=1$ and $\sigma=1.5$}
\label{fig:surf}
\end{figure}

\section*{5.2.1}


\end{document}
